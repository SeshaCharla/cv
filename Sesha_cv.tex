\documentclass[letterpaper,10pt]{article}

\usepackage{latexsym}
\usepackage[empty]{fullpage}
\usepackage{titlesec}
\usepackage{marvosym}
\usepackage[usenames,dvipsnames]{color}
\usepackage{verbatim}
\usepackage{enumitem}
\usepackage[hidelinks]{hyperref}
\usepackage{fancyhdr}
\usepackage[english]{babel}
\usepackage{tabularx}
\input{glyphtounicode}
%\usepackage[top=1cm, bottom=1cm, left=1cm, right=1cm]{geometry}
\usepackage[top=0.5in, bottom=0.5in, left=0.5in, right=0.5in]{geometry}
\usepackage{fancyhdr}
\usepackage[utf8]{inputenc}
\usepackage[english]{babel}

\raggedbottom
\raggedright

% Sections formatting
\titleformat{\section}{
  \vspace{-4pt}\scshape\raggedright\Large
}{}{0em}{}[\color{black}\titlerule \vspace{-5pt}]


\begin{document}
\begin{center}
\thispagestyle{empty}
\huge{\textit{Sesha N.} \textsc{Charla}}\\
\small PhD Candidate $\cdot$ School of Mechanical Engineering $\cdot$ Purdue University\\
\small scharla@purdue.edu $\mid$ +1 (765) 714 4235   \\
%\hrulefill
\end{center}



\begin{tabular}{l c l}
    {\large{\textsc{Objective}}}&:& Internship$/$Co-op.\\
	{\large{\textsc{Core Skills}}}&:& Control Engineering, System Identification, Signal Processing, Electromechanical Systems Design, \\
    & & Robotics, Embedded Programming, Machine Learning and Optimization.\\
\end{tabular}

%%%%%%%%%%%%%%%%%%%%%%%%%%%%%%%%%%%%%%%%%%%%%%%%%%%%%%%%%%%%%%%
% Work Experience
%%%%%%%%%%%%%%%%%%%%%%%%%%%%%%%%%%%%%%%%%%%%%%%%%%%%%%%%%%%%%%%

\section{Work Experience}
\noindent \textsc{Virgin Hyperloop} \hfill Los Angeles, California, USA.\\

\vspace{3pt}
\textbf{\textit{Motion Control Engineering Intern}} \hfill Jun. 2022 $-$ Aug. 2022
\begin{itemize}[noitemsep,nolistsep,leftmargin=0.25in,label={--}]
    \item Developed a supervisory control for minimizing the force missmatch
        accross the bogie for imporving eninge efficiency.
\end{itemize}

\vspace{6pt}
\noindent \textsc{U.R Rao Satellite centre (URSC) - Indian Space Research Organization (ISRO)} \hfill Benguluru, India \\

\vspace{3pt}
\textbf{\textit{Thermal Test Engineer [Scientist/Engineer "SC"], Thermal Systems Group}} \hfill Jul. 2016 $-$ Jul. 2019
\begin{itemize}[noitemsep,nolistsep,leftmargin=0.25in,label={--}]
    \item Worked thermal instrumentation and control for thermal-vacuum and thermal-balance tests of satellite systems.
	\item Team lead for \textit{Chandrayan-2} (the lunar lander project of ISRO) mission's orbiter, lander and rover.
    \item Augmented the software of existing \textit{SCADA} system for simultaneous testing of multiple satellites, doubling the productivity of test systems.
	\item Improved and standardized the heater design through a web application to streamline PI gain tuning and in-situ heater sensor mapping for the technology development program: \textit{Automation of thermal control during satellite tests}.
    \item Held the position of \textit{Project Manager: Thermal Testing, Human Space Program (HSP)} from Jan. $2019$ to Jul. $2019$, to assist in development of procedures and infrastructure for thermal vacuum and thermal balance testing of the proposed new designs of the spacecrafts.\\
\end{itemize}

\section{Academic Work Experience}
\noindent \textsc{Purdue University} \hfill West Lafayette, IN, USA \\
\vspace{3pt}
\noindent \textbf{\textit{Graduate Teaching Assistant}} \hfill Aug. 2020 $-$ Present\\
\begin{itemize}[noitemsep,nolistsep,leftmargin=0.25in,label={--}]
    \item Taught laboratory courses in Measurement and Control (ME 375, ME 365, AAE 364).
    \item Organized office hours for students to ask questions and discuss the course homeworks and projects.
\end{itemize}
\vspace{3pt}
\noindent \textbf{\textit{Graduate Research Assistant (Masters)}} \hfill Aug. 2020 $-$ Dec. 2020\\
\textit{Flight Dynamics, Controls and Hybrid Systems Lab}
\begin{itemize}[noitemsep,nolistsep,leftmargin=0.25in,label={--}]
	\item Developed a simplex chain design methodology using convex optimization for solving control problems specified using linear temporal logic (Reach Control Problem) in up to 3-dimensional state spaces with multiple control inputs and explored its possible applications to shared control and path planning.
\end{itemize}

%%%%%%%%%%%%%%%%%%%%%%%%%%%%%%%%%%%%%%%%%%%%%%%%%%%%%%%%%%%%%%%
% PUBLICATION
%%%%%%%%%%%%%%%%%%%%%%%%%%%%%%%%%%%%%%%%%%%%%%%%%%%%%%%%%%%%%%%
\section{Publications}
\begin{itemize}[noitemsep,nolistsep,leftmargin=0.25in,label={}]
\item Charla, S., Yao, B., Voyles, R., (2022). On Enhancing the Bandwidth
of the Actuator Dynamics in a Multi-rotor Aerial Vehicle. Manuscript submitted to 2022 Modeling, Estimation and Control Conference.
\end{itemize}


%%%%%%%%%%%%%%%%%%%%%%%%%%%%%%%%%%%%%%%%%%%%%%%%%%%%%%%%%%%%%%%
% EDUCATION
%%%%%%%%%%%%%%%%%%%%%%%%%%%%%%%%%%%%%%%%%%%%%%%%%%%%%%%%%%%%%%%
\section{Education}
\noindent \textsc{Purdue University}, \textit{School of Mechanical Engineering} \hfill West Lafayette, IN, USA \\
\textbf{\textit{PhD in Mechanical Engineering}} \hfill Aug. 2021 $-$ Present\\
\textsc{Area}: \textit{\textbf{Systems, Measurement and Controls}}  \hfill \textsc{Advisor}: Prof. Bin Yao\\
%\textsc{Course Work:} ME 586: Microprocessors in Electromechanical Systems, ME 584: System Identification.\hfill \textsc{GPA}: $3.81/4$\\
\textsc{Current Project}: Fault-tolerant controller design and implementation for a fully actuated tilted-wing hexacopter.\\

\vspace{5pt}
\noindent \textsc{Purdue University}, \textit{School of Aeronautics and Astronautics} \hfill West Lafayette, IN, USA \\
\textbf{\textit{Masters in Aeronautics and Astronautics}} \hfill Aug. 2019 $-$ Aug. 2021\\
\textsc{Major}: \textit{\textbf{Autonomy and Controls}}, \textsc{Minor}: \textit{\textbf{Machine Learning}}
%\hfill Advisor: Prof. Inseok Hwang\\
%\textsc{Course Work}: ME 686: Adaptive Control, AAE 668: Hybrid Systems, ME 677: Non-linear Systems, AAE 564: Linear Systems,  AAE 568: Optimal Control, ECE 680: Modern Automatic Control, CS 590: Reinforcement Learning, ECE 595: Machine Learning 1, STAT 545: Computational Statistics, MA 514 : Numerical Analysis
\hfill \textsc{GPA}: $3.77/4$\\


\vspace{5pt}
\noindent \textsc{Indian Institute of Space Science and Technology (IIST)} \hfill Thiruvananthapuram, India\\
\textbf{\textit{Bachelor of Technology in Aerospace Engineering}} \hfill \textsc{GPA}: $8.34/10$ \hfill Aug. 2012 $-$ May. 2016 \\
\textsc{Thesis Title}: Modelling and Simulation of a Flexible Launch Vehicle
\hfill \textsc{Adviser}: Prof. K. Kurien Issac \\
%\textsc{Major Project}: Design and Development of a Flapping Wing Micro Air Vehicle \hfill {Nov. 2013 $-$ Dec. 2014}\\
Recipient of \textit{DOS-ISRO Scholarship}.


%%%%%%%%%%%%%%%%%%%%%%%%%%%%%%%%%%%%%%%%%%%%%%%%%%%%%%%%%%%%%%%
% Technical Skills
%%%%%%%%%%%%%%%%%%%%%%%%%%%%%%%%%%%%%%%%%%%%%%%%%%%%%%%%%%%%%%%
\section{Technical Skills}
\begin{tabular}{l c l}
\textsc{Control Design}&:& Optimal Control, Robust and MIMO
Control, Adaptive and Nonlinear Control\\
	\textsc{Programming Languages}&:& Python, MATLAB, C, C++, Bash, R, Julia\\
    \textsc{Embedded Programming}&:& C, ARM Cortex-M3/M4 Assembly (TIVA-TM4C123GXL, STM32-Discovery)\\
	\textsc{OS \& Software Tools}&:& Linux, Git, Simulink, LabView, AutoCad, Catia, Solid Works, NX\\
	\textsc{Machining}&:& Lathe, Milling, Drilling; CNC programming\\
\end{tabular}
\end{document}
