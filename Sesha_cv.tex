\documentclass[letterpaper,10pt]{article}
\usepackage{latexsym}
\usepackage[empty]{fullpage}
\usepackage{titlesec}
\usepackage{marvosym}
\usepackage[usenames,dvipsnames]{color}
\usepackage{verbatim}
\usepackage{enumitem}
\usepackage[hidelinks]{hyperref}
\usepackage{fancyhdr}
\usepackage[english]{babel}
\usepackage{tabularx}
\input{glyphtounicode}
\usepackage[top=2.5cm, bottom=2.5cm, left=2cm, right=2cm]{geometry}
%\usepackage[top=1in, bottom=1in, left=1in, right=1in]{geometry}
\usepackage{fancyhdr}
\usepackage[utf8]{inputenc}
\usepackage[english]{babel}
\raggedbottom
\raggedright

% Sections formatting
\titleformat{\section}{
  \vspace{-4pt}\scshape\raggedright\Large
}{}{0em}{}[\color{black}\titlerule \vspace{-2pt}\titlerule \vspace{-5pt}]


\begin{document}
\begin{center}
\thispagestyle{empty}
\huge{\textit{Sesha N.} \textsc{Charla}}\\
\small PhD Candidate $\cdot$ School of Mechanical Engineering $\cdot$ Purdue University\\
\small scharla@purdue.edu $\mid$ +1 (765) 714 4235   \\
\end{center}


\begin{tabular}{l c l}
    {\large{\textsc{Objective}}}&:& Internship$/$Co-op.\\
{\large{\textsc{Core Skills}}}&:& Control Theory, Dynamic Modelling, System
Identification and Optimization\\
\end{tabular}

%%%%%%%%%%%%%%%%%%%%%%%%%%%%%%%%%%%%%%%%%%%%%%%%%%%%%%%%%%%%%%%
% Work Experience
%%%%%%%%%%%%%%%%%%%%%%%%%%%%%%%%%%%%%%%%%%%%%%%%%%%%%%%%%%%%%%%

\section{Work Experience}
\noindent \textsc{Virgin Hyperloop} \hfill Los Angeles, California, USA.\\
\vspace{3pt}
\textbf{\textit{Motion Control Engineering Intern}} \hfill Jun. 2022 $-$ Aug. 2022
\begin{itemize}[noitemsep,nolistsep,leftmargin=0.25in,label={--}]
    \item Developed a supervisory control for minimizing the force mismatch
across the bogie for improving EM engine efficiency and worked out the necessary
requirements through cross team meetings.
\item Refined the control architecture for single EM engine control and
demonstrated the limitation of the linear design on the non-linear system.
\item Developed a MIMO model for the guidance bogie engine assembly and worked
out the stability of off-diagonal elements w.r.t suspension parameters.
\end{itemize}
%===============================================================================
%===============================================================================
\vspace{10pt}
\noindent \textsc{U.R Rao Satellite centre (URSC) - Indian Space Research Organization (ISRO)} \hfill India \\
\vspace{3pt}
\textbf{\textit{Thermal Test Engineer, Thermal Systems Group}} \hfill Jul. 2016 $-$ Jul. 2019
\begin{itemize}[noitemsep,nolistsep,leftmargin=0.25in,label={--}]
    \item Worked thermal instrumentation and control for thermal-vacuum and thermal-balance tests of satellite systems.
	\item Team lead for \textit{Chandrayan-2} (the lunar lander project of ISRO) mission's orbiter, lander and rover.
    \item Augmented the software of existing \textit{SCADA} system for simultaneous testing of multiple satellites, doubling the productivity of test systems.
	\item Improved and standardized the heater design through a web application to streamline PI gain tuning and in-situ heater sensor mapping for the technology development program: \textit{Automation of thermal control during satellite tests}.
    \item Held the position of \textit{Project Manager, Thermal Testing, Human Space Program (HSP)} from Jan. $2019$ to Jul. $2019$, to assist in development of procedures and infrastructure for thermal vacuum and thermal balance testing of the proposed new designs of the spacecrafts.\\
\end{itemize}

%%%%%%%%%%%%%%%%%%%%%%%%%%%%%%%%%%%%%%%%%%%%%%%%%%%%%%%%%%%%%%%
% PUBLICATION
%%%%%%%%%%%%%%%%%%%%%%%%%%%%%%%%%%%%%%%%%%%%%%%%%%%%%%%%%%%%%%%
\renewcommand\refname{Publications}
\nocite{improvBandwidth}
\bibliographystyle{unsrt}
\bibliography{ref}
%%%%%%%%%%%%%%%%%%%%%%%%%%%%%%%%%%%%%%%%%%%%%%%%%%%%%%%%%%%%%%%
% EDUCATION
%%%%%%%%%%%%%%%%%%%%%%%%%%%%%%%%%%%%%%%%%%%%%%%%%%%%%%%%%%%%%%%
\section{Education}
\noindent \textsc{Purdue University}, \textit{School of Mechanical Engineering} \hfill West Lafayette, IN, USA \\
\textbf{\textit{PhD in Mechanical Engineering}} \hfill Aug. 2021 $-$ Present\\
\textsc{Area}: \textit{\textbf{Systems, Measurement and Controls}}  \hfill \textsc{Advisor}: Prof. Bin Yao\\
%\textsc{Course Work:} ME 586: Microprocessors in Electromechanical Systems, ME
%584: System Identification, ME 675: Multivariable Control.
\textsc{Current Project}: Fault-tolerant controller design and implementation on
a fully actuated tilted-wing hexacopter for aerial manipulation.\hfill \textsc{GPA}: $3.75/4$\\
\vspace{5pt}
\noindent \textsc{Purdue University}, \textit{School of Aeronautics and Astronautics} \hfill West Lafayette, IN, USA \\
\textbf{\textit{Masters in Aeronautics and Astronautics}} \hfill Aug. 2019 $-$ Aug. 2021\\
\textsc{Major}: \textit{\textbf{Autonomy and Controls}}, \textsc{Minor}: \textit{\textbf{Machine Learning}}
%\hfill Advisor: Prof. Inseok Hwang\\
%\textsc{Course Work}: ME 686: Adaptive Control, AAE 668: Hybrid Systems, ME 677: Non-linear Systems, AAE 564: Linear Systems,  AAE 568: Optimal Control, ECE 680: Modern Automatic Control, CS 590: Reinforcement Learning, ECE 595: Machine Learning 1, STAT 545: Computational Statistics, MA 514 : Numerical Analysis
\hfill \textsc{GPA}: $3.77/4$\\

\vspace{5pt}
\noindent \textsc{Indian Institute of Space Science and Technology (IIST)} \hfill Thiruvananthapuram, India\\
\textbf{\textit{Bachelor of Technology in Aerospace Engineering}} \hfill \textsc{GPA}: $8.34/10$ \hfill Aug. 2012 $-$ May. 2016 \\
\textsc{Thesis}: Finite Element Dynamic Model using Lagrangian for Launch
Vehicle Bending and Trajectory Simulations.
Recipient of \textit{DOS-ISRO Scholarship}.
\hfill \textsc{Adviser}: Prof. K. Kurien Issac \\

%===============================================================================
% Academic Work Experience
%===============================================================================
\section{Academic Work Experience}
\noindent \textsc{Purdue University} \hfill West Lafayette, IN, USA \\
\vspace{3pt}
\noindent \textbf{\textit{Graduate Teaching Assistant}} \hfill Aug. 2020 $-$ Present\\
\begin{itemize}[noitemsep,nolistsep,leftmargin=0.25in,label={--}]
\item Teaching Assistant for Microprocessors and Electromechanical Systems $(ME
586)$ course and Programming Lab.
    \begin{itemize}
\item Developing control system for refrigeration system with variable speed
compressor and flow control valve and implementing it using STM32 Discovery
board (guided final project for the course).
\item Conducting laboratory and office hours for assembly and C programming on
STM32 Discovery for implementing control algorithms.
    \end{itemize}
\item Taught undergraduate laboratory courses in Measurement and Control (ME
375, ME 365, AAE 364).
\end{itemize}
\vspace{3pt}
\noindent \textbf{\textit{Graduate Research Assistant (Masters)}} \hfill Aug. 2020 $-$ Dec. 2020\\
\textit{Flight Dynamics, Controls and Hybrid Systems Lab}
\begin{itemize}[noitemsep,nolistsep,leftmargin=0.25in,label={--}]
	\item Developed a simplex chain design methodology using convex optimization for solving control problems specified using linear temporal logic (Reach Control Problem) in up to 3-dimensional state spaces with multiple control inputs and explored its possible applications to shared control and path planning.
\end{itemize}



%%%%%%%%%%%%%%%%%%%%%%%%%%%%%%%%%%%%%%%%%%%%%%%%%%%%%%%%%%%%%%%
% Technical Skills
%%%%%%%%%%%%%%%%%%%%%%%%%%%%%%%%%%%%%%%%%%%%%%%%%%%%%%%%%%%%%%%
\section{Technical Skills}
\begin{tabular}{l c l}
\textsc{Control Design}&:& Robust control, MIMO systems, Adaptive and Nonlinear control\\
	\textsc{Programming Languages}&:& Python, MATLAB, C, C++, Bash, R, Julia\\
    \textsc{Embedded Programming}&:& C, C++, ARM Cortex-M3/M4 Assembly, LabView for NI-MyRio\\
	\textsc{OS \& Software Tools}&:& Linux, Git, Simulink, Jira, \LaTeX\\
	%\textsc{Machining}&:& Lathe, Milling, Drilling; CNC programming\\
\end{tabular}
\end{document}
