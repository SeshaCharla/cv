\documentclass[letterpaper,10pt]{article}

\usepackage{latexsym}
\usepackage[empty]{fullpage}
\usepackage{titlesec}
\usepackage{marvosym}
\usepackage[usenames,dvipsnames]{color}
\usepackage{verbatim}
\usepackage{enumitem}
\usepackage[hidelinks]{hyperref}
\usepackage{fancyhdr}
\usepackage[english]{babel}
\usepackage{tabularx}
\input{glyphtounicode}
%\usepackage[top=1cm, bottom=1cm, left=1cm, right=1cm]{geometry}
\usepackage[top=0.5in, bottom=0.5in, left=0.5in, right=0.5in]{geometry}
\usepackage{fancyhdr}
\usepackage[utf8]{inputenc}
\usepackage[english]{babel}

\raggedbottom
\raggedright

% Sections formatting
\titleformat{\section}{
  \vspace{-4pt}\scshape\raggedright\Large
}{}{0em}{}[\color{black}\titlerule \vspace{-5pt}]


\begin{document}
\begin{center}
\thispagestyle{empty}
\huge{\textit{Sesha N.S.L Sai} \textsc{Charla}}\\
\small Graduate Student $\cdot$ School of Aeronautics and Astronautics $\cdot$ Purdue University\\
\small scharla@purdue.edu $\mid$ +1 (765) 714 4235   \\
%\hrulefill
\end{center}



\begin{tabular}{l c l}
	{\large{\textsc{Objective}}}&:&Pursue PhD in Automatic Controls and, ultimately, a career in research and development.\\
	{\large{\textsc{Core Skills}}}&:& Control Systems Design and Analysis \& Technology Development and Integration. %and Project Management\\
\end{tabular}

%%%%%%%%%%%%%%%%%%%%%%%%%%%%%%%%%%%%%%%%%%%%%%%%%%%%%%%%%%%%%%%
% Work Experience
%%%%%%%%%%%%%%%%%%%%%%%%%%%%%%%%%%%%%%%%%%%%%%%%%%%%%%%%%%%%%%%

\section{Work Experience} 
\noindent \textbf{Purdue University} \hfill West Lafayette, IN, USA \\
\noindent \textit{Graduate Research Assistant, Flight Dynamics, Controls and Hybrid Systems Lab} \hfill Aug. 2020 $-$ present
\begin{itemize}[noitemsep,nolistsep,leftmargin=0.25in,label={--}]
    \item Worked on research problem formulation, quarterly report to funding agency and monthly presentations to the collaborators regarding the project: "Cyber-physical security of Urban Air Mobility (UAM) systems".
    \item Also worked on "Human Machine Mutual Adaptation Project" alongside the thesis work on "Simplex Chain Design for Reach Control".
\end{itemize}

\vspace{3pt}
\noindent \textit{Graduate Teaching Assistant, AAE 421: Flight Dynamics and Controls by prof. Ran Dai} \hfill Aug. 2020 $-$ Dec 2020
\begin{itemize}[noitemsep,nolistsep,leftmargin=0.25in,label={--}]
    \item Prepared homework and exam solutions, coordinated grading with one another TA and a grader and organized office hours for explaining the concepts to students.
\end{itemize}

\vspace{5pt}
\noindent \textbf{U.R Rao Satellite centre (URSC) - Indian Space Research Organization (ISRO)} \hfill Benguluru, India \\
\textit{Project Manager: Thermal Testing, Human Space Program (HSP)} \hfill Jan. 2019 $-$ Jul. 2019
\begin{itemize}[noitemsep,nolistsep,leftmargin=0.25in,label={--}]
    \item Organized meetings to finalize test procedures and necessary infrastructure for thermal vacuum and thermal balance testing of the proposed designs of the spacecraft.
\end{itemize}

\vspace{3pt}
\textit{Test Engineer [Scientist/Engineer "SC"], Thermal Systems Group} \hfill Jul. 2016 $-$ Jul. 2019
\begin{itemize}[noitemsep,nolistsep,leftmargin=0.25in,label={--}]
    \item Conducted Thermal Vacuum and Thermal Balance tests of $25$ \textit{satellites} in 3 years as a part of a team of 8 engineers in collaborations with ground checkout teams.
	\item \textit{Led a team} of 4 engineers and 2 technicians for Thermal Vacuum and Thermal Balance tests of \textit{Chandrayan-2} (the lunar lander project of ISRO) mission's orbiter, lander and rover. 
    \item Augmented the software of existing \textit{SCADA} systems for simultaneous testing of multiple satellites.  
	\item Started a technology development program: \textit{Automation of thermal control during satellite tests}
	\item Involved in technology development programs: $(1)$ Structural analysis of a newly designed \textit{Passive Radiative Cooler}. $(2)$ DSMC method for Aerobraking.
\end{itemize}



%%%%%%%%%%%%%%%%%%%%%%%%%%%%%%%%%%%%%%%%%%%%%%%%%%%%%%%%%%%%%%%
% EDUCATION
%%%%%%%%%%%%%%%%%%%%%%%%%%%%%%%%%%%%%%%%%%%%%%%%%%%%%%%%%%%%%%%
\section{Education}
\noindent \textbf{Purdue University}, \textit{School of Aeronautics and Astronautics} \hfill West Lafayette, IN, USA \\
\textit{Masters in Aeronautics and Astronautics} \hfill Aug. 2019 $-$ May. 2021 (Anticipated)\\
\textsc{Major}: \textit{\textbf{Autonomy and Controls}}, \textsc{Minor}: \textit{\textbf{Machine Learning}} \hfill \textsc{Adviser}: Prof. Inseok Hwang\\
\textsc{Course Work}: AAE 668: Hybrid Systems, ME 677: Non-linear Systems, AAE 564: Linear Systems,  AAE 568: Optimal Control, ECE 680: Modern Automatic Control, CS 590: Reinforcement Learning, ECE 595: Machine Learning 1, STAT 545: Computational Statistics
\hfill \textsc{GPA}: $3.84/4$\\
\textsc{Thesis Title}: Simplex Chain Design for Reach Control \\

\vspace{5pt}
\noindent \textbf{Indian Institute of Science}, \textit{Center for Continuing Education} \hfill Jan. 2017 $-$ May. 2018  \hfill Bengaluru, India \\
\textsc{Course Work:} Embedded Systems on ARM, Structural Optimization and Vibration and Noise Control \\

\vspace{5pt}
\noindent \textbf{Indian Institute of Space Science and Technology (IIST)} \hfill Thiruvananthapuram, India\\
\textit{Bachelor of Technology in Aerospace Engineering} \hfill \textsc{GPA}: $8.34/10$ \hfill Aug. 2012 $-$ May. 2016 \\
\textsc{Thesis Title}: Modelling and Simulation of a Flexible Launch Vehicle
\hfill \textsc{Adviser}:Prof. K. Kurien Issac \\
\textsc{Major Projects}: $(1)$ Control of an Under-Actuated System: Torque Limited Pendulum \hfill {Summer 2015}\\
$(2)$ Design and Development of a Flapping Wing Micro Air Vehicle \hfill {Nov. 2013 $-$ Dec. 2014}\\
Recipient of \textit{DOS-ISRO Scholarship}.


%%%%%%%%%%%%%%%%%%%%%%%%%%%%%%%%%%%%%%%%%%%%%%%%%%%%%%%%%%%%%%%
% Technical Skills
%%%%%%%%%%%%%%%%%%%%%%%%%%%%%%%%%%%%%%%%%%%%%%%%%%%%%%%%%%%%%%%
\section{Technical Skills}
\begin{tabular}{l c l}
	\textsc{Programming Languages}&:& Python, Bash, R, MATLAB,C, C++ \\
	\textsc{OS \& Software Tools}&:& Linux, Git, vim, \LaTeX, MS Office\\
    \textsc{Drafting and CAD Tools}&:& AutoCad, Catia, Solid Works, NX\\
    \textsc{Development Boards}&:& ARM Cortex M3/M4 (TM4C123GXL), Arduino, Raspberry PI\\
	\textsc{Manufacturing}&:& Machining - Lathe, Milling, Drilling, Grinding,
	Welding\\
\end{tabular}
\end{document}
